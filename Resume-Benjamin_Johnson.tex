\documentclass[10pt]{article}

% ------------------------------------------------------------------------------

% Resume - Benjamin Johnson

% This is my personal resume, used for job applications and such.

% ------------------------------------------------------------------------------

% This code is pretty messy at the moment.

% ------------------------------------------------------------------------------

% This is a helpful package that puts math inside length specifications
\usepackage{calc}

% Simpler bibsection for CV sections
% (thanks to natbib for inspiration)
\makeatletter
\newlength{\bibhang}
\setlength{\bibhang}{1em}
\newlength{\bibsep}
{\@listi \global\bibsep\itemsep \global\advance\bibsep by\parsep}
\newenvironment{bibsection}%
{\vspace{-\baselineskip}\begin{list}{}{%
      \setlength{\leftmargin}{\bibhang}%
      \setlength{\itemindent}{-\leftmargin}%
      \setlength{\itemsep}{\bibsep}%
      \setlength{\parsep}{\z@}%
      \setlength{\partopsep}{0pt}%
      \setlength{\topsep}{0pt}}}
  {\end{list}\vspace{-.6\baselineskip}}
\makeatother

% Layout: Puts the section titles on left side of page
\reversemarginpar

%
%         PAPER SIZE, PAGE NUMBER, AND DOCUMENT LAYOUT NOTES:
%
% The next \usepackage line changes the layout for CV style section
% headings as marginal notes. It also sets up the paper size as either
% letter or A4. By default, letter was used. If A4 paper is desired,
% comment out the letterpaper lines and uncomment the a4paper lines.
%
% As you can see, the margin widths and section title widths can be
% easily adjusted.
%
% ALSO: Notice that the includefoot option can be commented OUT in order
% to put the PAGE NUMBER *IN* the bottom margin. This will make the
% effective text area larger.
%
% IF YOU WISH TO REMOVE THE ``of LASTPAGE'' next to each page number,
% see the note about the +LP and -LP lines below. Comment out the +LP
% and uncomment the -LP.
%
% IF YOU WISH TO REMOVE PAGE NUMBERS, be sure that the includefoot line
% is uncommented and ALSO uncomment the \pagestyle{empty} a few lines
% below.
%

%% Use these lines for letter-sized paper
\usepackage[paper=letterpaper,
            %includefoot, % Uncomment to put page number above margin
            marginparwidth=2.2cm,     % Length of section titles
            marginparsep=0.1cm,       % Space between titles and text
            margin=2cm,               % 1 inch margins
            includemp]{geometry}

%% Use these lines for A4-sized paper
%\usepackage[paper=a4paper,
%            %includefoot, % Uncomment to put page number above margin
%            marginparwidth=30.5mm,    % Length of section titles
%            marginparsep=1.5mm,       % Space between titles and text
%            margin=25mm,              % 25mm margins
%            includemp]{geometry}

%% More layout: Get rid of indenting throughout entire document
\setlength{\parindent}{0in}

%% This gives us fun enumeration environments. compactitem will be nice.
\usepackage{paralist}

%% Reference the last page in the page number
%
% NOTE: comment the +LP line and uncomment the -LP line to have page
%       numbers without the ``of ##'' last page reference)
%
% NOTE: uncomment the \pagestyle{empty} line to get rid of all page
%       numbers (make sure includefoot is commented out above)
%
\usepackage{fancyhdr,lastpage}
\pagestyle{fancy}
\pagestyle{empty}      % Uncomment this to get rid of page numbers
\fancyhf{}\renewcommand{\headrulewidth}{0pt}
\fancyfootoffset{\marginparsep+\marginparwidth}
\newlength{\footpageshift}
\setlength{\footpageshift}
          {0.5\textwidth+0.5\marginparsep+0.5\marginparwidth-2in}
\lfoot{\hspace{\footpageshift}%
       \parbox{4in}{\, \hfill %
                    \arabic{page} of \protect\pageref*{LastPage} % +LP
%                    \arabic{page}                               % -LP
                    \hfill \,}}

% Finally, give us PDF bookmarks
\usepackage{color,hyperref}
\definecolor{darkblue}{rgb}{0.0,0.0,0.3}
\hypersetup{colorlinks,breaklinks,
            linkcolor=darkblue,urlcolor=darkblue,
            anchorcolor=darkblue,citecolor=darkblue}

%%%%%%%%%%%%%%%%%%%%%%%% End Document Setup %%%%%%%%%%%%%%%%%%%%%%%%%%%%


%%%%%%%%%%%%%%%%%%%%%%%%%%% Helper Commands %%%%%%%%%%%%%%%%%%%%%%%%%%%%

% The title (name) with a horizontal rule under it
%
% Usage: \makeheading{name}
%
% Place at top of document. It should be the first thing.
\newcommand{\makeheading}[1]%
        {\hspace*{-\marginparsep minus \marginparwidth}%
         \begin{minipage}[t]{\textwidth+\marginparwidth+\marginparsep}%
                {\Large \bfseries #1}\\[-0.8\baselineskip]%
                 \rule{\columnwidth}{0.25pt}%
         \end{minipage}}

% The section headings
%
% Usage: \section{section name}
%
% Follow this section IMMEDIATELY with the first line of the section
% text. Do not put whitespace in between. That is, do this:
%
%       \section{My Information}
%       Here is my information.
%
% and NOT this:
%
%       \section{My Information}
%
%       Here is my information.
%
% Otherwise the top of the section header will not line up with the top
% of the section. Of course, using a single comment character (%) on
% empty lines allows for the function of the first example with the
% readability of the second example.
\renewcommand{\section}[2]%
        {\pagebreak[3]\vspace{1.3\baselineskip}%
         \phantomsection\addcontentsline{toc}{section}{#1}%
         \hspace{0in}%
         \marginpar{
         \raggedright \scshape #1}#2}

% An itemize-style list with lots of space between items
\newenvironment{outerlist}[1][\enskip\textbullet]%
        {\begin{itemize}[#1]}{\end{itemize}%
         \vspace{-.6\baselineskip}}

% An environment IDENTICAL to outerlist that has better pre-list spacing
% when used as the first thing in a \section
\newenvironment{lonelist}[1][\enskip\textbullet]%
        {\vspace{-\baselineskip}\begin{list}{#1}{%
        \setlength{\partopsep}{0pt}%
        \setlength{\topsep}{0pt}}}
        {\end{list}\vspace{-.6\baselineskip}}

% An itemize-style list with little space between items
\newenvironment{innerlist}[1][\enskip\textbullet]%
        {\begin{compactitem}[#1]}{\end{compactitem}}

% An environment IDENTICAL to innerlist that has better pre-list spacing
% when used as the first thing in a \section
\newenvironment{loneinnerlist}[1][\enskip\textbullet]%
        {\vspace{-\baselineskip}\begin{compactitem}[#1]}
        {\end{compactitem}\vspace{-.6\baselineskip}}

% To add some paragraph space between lines.
% This also tells LaTeX to preferably break a page on one of these gaps
% if there is a needed pagebreak nearby.
\newcommand{\blankline}{\quad\pagebreak[2]\vspace{-0.3\baselineskip}}

% For \url{SOME_URL}, links SOME_URL to the url SOME_URL
\providecommand*\url[1]{\href{#1}{#1}}
% Same as above, but pretty-prints SOME_URL in teletype fixed-width font
\renewcommand*\url[1]{\href{#1}{\texttt{#1}}}

% For \email{ADDRESS}, links ADDRESS to the url mailto:ADDRESS
\providecommand*\email[1]{\href{mailto:#1}{#1}}
% Same as above, but pretty-prints ADDRESS in teletype fixed-width font
%\renewcommand*\email[1]{\href{mailto:#1}{\texttt{#1}}}

%%%%%%%%%%%%%%%%%%%%%%%% End Helper Commands %%%%%%%%%%%%%%%%%%%%%%%%%%%

%%%%%%%%%%%%%%%%%%%%%%%%% Begin CV Document %%%%%%%%%%%%%%%%%%%%%%%%%%%%

% ------------------------------------------------------------------------------

%% Multilingual Support
\usepackage{polyglossia}
\setdefaultlanguage{english}
%\setotherlanguage{spanish}

%% Typeface choices
\usepackage{fontspec}

% We want to use ligatures by default in text fonts, but math and mono fonts may
% lack these features (for good reason), so we limit our scope.
\defaultfontfeatures[\rmfamily, \sffamily]{Ligatures={Common,TeX}}

% Used by any language without another font defined
% \setmainfont{Linux Libertine O}
% \setsansfont{Linux Biolinum O}
% \setmonofont{Inconsolata}
% \setmainfont{Source Serif Pro}
% \setsansfont{Source Sans Pro}
% \setmonofont{Source Code Pro}
\setmainfont{TeX Gyre Pagella}
\setsansfont{TeX Gyre Heros}
\setmonofont{Inconsolata}

% English
% Used as default language for the document
%\newfontfamily\englishfont[Script=Latin]{Noto Serif}

% ------------------------------------------------------------------------------

\usepackage[autostyle=true]{csquotes}

\usepackage{setspace}

% For units
\usepackage[binary-units]{siunitx}

% ------------------------------------------------------------------------------

% PACKAGES for TABLES and GRAPHICS

% Allows use of extra options on \includegraphics
\usepackage{graphicx}
% Allows use of \enumerate[...]
\usepackage{enumerate}
% Provides \multirow
\usepackage{multirow}

\usepackage{array}

% PACKAGES for COLORS

\usepackage[usenames,dvipsnames,svgnames,table]{xcolor}

% PACKAGES for HYPERLINKS and METADATA

\usepackage{hyperref}
\usepackage{hyperxmp}

\hypersetup{
  colorlinks={true},
  %linkcolor={black},
  %urlcolor={black},
  pdfauthor={Benjamin Johnson}
}

% ------------------------------------------------------------------------------

% My own macros for consistent typesetting of certain nouns / abbreviations

\newcommand{\EASA}{\textsc{Easa}}
\newcommand{\easa}{\EASA}

\newcommand{\MATLAB}{\textsc{Matlab}}
\newcommand{\matlab}{\MATLAB}
\newcommand{\IGOR}{\textsc{Igor} Pro}
\newcommand{\igor}{\IGOR}

\newcommand{\html}{\textsc{Html}}
\newcommand{\css}{\textsc{Css}}
\newcommand{\php}{\textsc{Php}}

\newcommand{\cpp}{C\nolinebreak\hspace{-.05em}\raisebox{.2ex}
  {\small +}\nolinebreak\hspace{-.10em}\raisebox{.2ex}{\small +}}
%\newcommand{\cpp}{C++}

\newcommand{\gdb}{\textsc{Gdb}}
% \newcommand{\svn}{\textsc{svn}}

\newcommand{\risc}{\textsc{Risc}}

\newcommand{\apbs}{\textsc{Apbs}}
\newcommand{\cuda}{\textsc{Cuda}}

\newcommand{\boinc}{\textsc{Boinc}}
\newcommand{\slurm}{\textsc{Slurm}}

\newcommand{\ctan}{\textsc{Ctan}}
\newcommand{\gnu}{\textsc{Gnu}}

%\usepackage{tabularx}

\begin{document}
\makeheading{Benjamin Johnson}

\section{Contact Information}
%
% NOTE: Mind where the & separators and \\ breaks are in the following
%       table.
%
% ALSO: \rcollength is the width of the right column of the table
%       (adjust it to your liking; default is 1.85in).
%
%\newlength{\rcollength}\setlength{\rcollength}{1.75in}
%
%\begin{tabular}[t]
%  {@{}p{\textwidth-\rcollength-2.3in}p{2.3in}p{\rcollength}}
\begin{tabular}[t]
  {@{}p{4cm}>{\centering\arraybackslash}p{\textwidth-9.5cm}p{5.5cm}}
  Harvey Mudd College    & OpenPGP Fingerprint & \href{tel:3477626467}{+1\,347\,762\,6467} \\
  340 East Foothill Blvd & \texttt{429C 43B8 94F7 67B4 D167} & \href{mailto:mangorune@gmail.com}{mangorune@gmail.com} \\
  Claremont, CA 91711    & \texttt{D46C E50C F045 9621 433F} & \href{https://www.linkedin.com/in/mangorune}{linkedin.com/in/mangorune}
\end{tabular}




% \section{Objective}
% This is silly

\section{Education}
    \textbf{Harvey Mudd College}, Claremont, California
    \hfill \textbf{Expected Graduation: May 2016} \\
    \textit{Physics Major}
    %\begin{innerlist}
    %\item Current GPA: 2.84
    %\item

      \begin{tabular}{ccc}
        \multicolumn{3}{c}{\textsc{Selected Coursework}} \\
        Data Structures \& Program Dev. &
        Discrete Mathematics &
        Theoretical Mechanics \\
        Computational Biology &
        Intermediate Linear Algebra &
        Quantum Mechanics \\
        Adv. Topics in Algorithms &
        Mathematical Analysis I &
        Quantum Information
      \end{tabular}
      % Computability and Logic
      % Computer Systems

      % Differential Equations and Linear Algebra II
      % Multivariable Calculus,

      % Electromagnetic Theory \& Optics,
      % Computational Methods in Physics
    %\end{innerlist}

    % \blankline

    % \textbf{McMinnville High School}, McMinnville, Oregon
    % \hfill \textbf{June 2011}
    % \begin{innerlist}
    % \item Cumulative GPA: 4.0; Valedictorian
    % % \item AP Test score of 5 in: Calculus BC, Computer Science A,
    % %   Biology, Physics C, and US History.
    % \item Engineeing and Aerospace Sciences Academy
    %   (\EASA{}), 2008--2010
    % \end{innerlist}

\section{Software Skills}
    Proficient: \LaTeX, Python, C/\cpp{}, Java. \\
    Familiar: *nix Shell Scripting, Mathematica, \igor{}, \matlab{},
    \html{}.

\section{Work Experience}
    \textbf{Software Engineering Intern}
    \hfill \textbf{May 2014 -- August 2014} \\
    \textit{Google, Mountain View, California} \\
    Worked on the web rendering pipeline within the Knowledge: Search
    Infrastructure group.
    \begin{innerlist}
    \item Wrote the foundation of a new load-management framework for all
      back-end web rendering.
    % These two didn't really make enough progress to be worth
    % mentioning.
    % \item Designing stochastic, parallel version of Latent Dirichlet
    %   Allocation (LDA) for topic modeling on an internal \boinc{} grid.
    % \item Tying Adaptive Poisson-Boltzmann Solver (\apbs{})
    %   solvation calculations into \cuda{} libraries.
    \end{innerlist}

    \blankline

    \textbf{Computer Science Grader and Tutor}
    \hfill \textbf{September 2012 -- December 2013} \\
    \textit{Harvey Mudd College, Claremont, California}

    \blankline

    \textbf{Technical Intern, Level 3}
    \hfill \textbf{May 2013 -- August 2013} \\
    \textit{Pacific Northwest National Laboratory,
      Richland, Washington} \\
    Worked on social media analytics and algorithm development in the
    Knowledge Discovery and Informatics group
    (\href{http://kdi.pnnl.gov/}{kdi.pnnl.gov}). Sponsored by the
    National Security Internship Program
    (\href{http://science-ed.pnnl.gov/nsip/}{science-ed.pnnl.gov/nsip}).
    \begin{innerlist}
    \item Rebuilt corrupt \SI{12}{\tebi\byte} document index from
      \SI{83}{\tebi\byte} raw data store.
    \item Developed flexible load balancer for social media search
      framework running on a \slurm{} cluster.
    % These two didn't really make enough progress to be worth
    % mentioning.
    % \item Designing stochastic, parallel version of Latent Dirichlet
    %   Allocation (LDA) for topic modeling on an internal \boinc{} grid.
    % \item Tying Adaptive Poisson-Boltzmann Solver (\apbs{})
    %   solvation calculations into \cuda{} libraries.
    \end{innerlist}

    \blankline

    \textbf{Student Researcher}
    \hfill \textbf{June 2012 -- July 2012} \\
    \textit{Harvey Mudd College, Claremont, California} \\
    Worked independently to set up and operate a real-time system for
    monitoring local atmospheric levels of light-absorbing,
    water-soluble organic aerosol. For more information on the Hawkins
    Lab:
    \href{http://newwww.hmc.edu/hawkinslab/LeliaHawkins_HMC/Welcome.html}
    {hmc.edu/hawkinslab}

    % \blankline

    % \textbf{Chemistry Stockroom Assistant}
    % \hfill \textbf{September 2011 -- December 2011} \\
    % \textit{Harvey Mudd College, Claremont, California}

\section{Other Project Experience}
    \textbf{Server Administrator}
    \hfill \textbf{May 2012 -- Present} \\
    \textit{Claremont, California} \\
    Independent project providing general-purpose shared storage for the
    school community as well as local mirrors of external resources. Took over
    from a graduating senior.
    \begin{innerlist}
    \item Reconfigured Arch Linux installation to utilize recent system
      improvements like systemd.
    \item Extended available storage space from \SI{9}{\tebi\byte} to
      \SI{36}{\tebi\byte}.
    \item Added mirrors of open source project repositories including Arch
      Linux, \ctan{}, \gnu{}, Apache, English Wikipedia, and Wikileaks.
    \end{innerlist}

    \blankline

    \textbf{Data Structures and Program Development}
    % CS 70 "Data Structures / Program Development
    \hfill \textbf{January 2012 -- May 2012} \\
    \textit{Harvey Mudd College, Claremont, California} \\
    Software development course in \cpp{}
    utilizing Subversion, Valgrind, and \gdb{}.
    Projects were pair-programmed and included
    a \risc{} memory cache simulator,
    a spell checker,
    a learning rock-paper-scissors player,
    refactoring a genetic algorithm,
    and implementations of BigInt, HashSet, and TreeSet.

    % \blankline

    % \textbf{Hovercraft}
    % \hfill \textbf{November 2009 -- April 2010} \\
    % \textit{Engineering and Aviation Sciences Academy, McMinnville,
    %   Oregon} \\
    % Built a hovercraft with a team of fellow students.

    % \blankline

    % \textbf{Helicopter Simulator}
    % \hfill \textbf{2008 -- 2009} \\ % I don't remember which months
    % \textit{Engineering and Aviation Sciences Academy, McMinnville,
    %   Oregon} \\
    % Worked on a team of students and adult mentors reverse engineering
    % and repairing a helicopter simulator which had fallen into
    % disrepair.

\section{Activities}
    \textbf{Windward Intercollegiate Code War}
    \hfill \textbf{January 2012, 2013} \\
    Worked in a team of five to implement a solution to an artificial
    intelligence challenge in 8 hours.
    Solutions were run against competing teams
    from other colleges across the country.

% \section{References}
%     \begin{tabular}[t]{@{}p{3.0in}p{3.0in}}
%     \textbf{Dr Lelia Hawkins}               & \textbf{Dr Christopher Stone}     \\
%     Assistant Professor, Chemistry          & Associate Professor, Computer Science \\
%     Harvey Mudd College                     & Harvey Mudd College               \\
%     phone: (909) 621-8522                   & phone: (909) 607-8975             \\
%     email: lelia\_hawkins@hmc.edu            & email: stone@cs.hmc.edu           \\
%     % ~                                       &   \\
%     % \textbf{Dr Tom Dietterich}              & \textbf{Dr Susan Martonosi}       \\
%     % Professor, Computer Science             & Assistant Professor, Math         \\
%     % Oregon State University                 & Harvey Mudd College               \\
%     % phone: (541) 737-5539                   & phone: (909) 607-0481             \\
%     % email: tgd@cs.orst.edu                  & email: martonosi@math.hmc.edu     \\
%     \end{tabular}

\end{document}
